
\noindent 
\textbf{\stepcounter{zadatak}
\thecjelina.\thezadatak.}
Materijalna točka (MT)  giba se u xy-ravnini tako da joj se vektor položaja mijenja u vremenu prema izrazu
$$
%\vec{r}(t)=t\mathrm{e}^{-2t}\vec{i}+\sqrt{t}\vec{j}\ [m].
\vec{r}(t)=\sqrt[5]{t}\vec{i}+t^2\cos(3t)\vec{j}\ [m]. 
$$
Koliki je iznos trenutnog ubrzanja materijalne točke u trenutku $t_1=0,3\ s$?

%\begin{enumerate}[label=\alph*)]
% \item Vektor i iznos trenutne brzine MT u trenutku $t=0,3\ s$.
% \item  Vektor i iznos trenutnog ubrzanja MT u trenutku $t=0,3\ s$.
% \end{enumerate}
