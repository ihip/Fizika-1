
\noindent 
\textbf{\stepcounter{zadatak}
\thecjelina.\thezadatak.}
Gibanje materijalne točke (MT) opisano je vektorom položaja
$$
\vec{r}(t)=(v_0 t)\vec{j} +(z_0 -\frac{1}{2}gt^2)\vec{k}.
$$
U trenutku $t=0\ s$ MT se nalazi na visini $z_0=80\ m$, a iznos početne brzine je $v_0=30\ ms^{-1}$.
Iznos ubrzanja slobodnog pada je $g=9,81\ ms^{-2}$, ali radi lakšek računanja može se uzeti približna vrijednost $g=10\ ms^{-2}$.
\begin{enumerate}[label=\alph*)]
  \item Izračunajte položaj MT svakih pola sekunde i skicirajte putanju u $yz$-ravnini.
  \item  Odredite vektor trenutne brzine $\vec{v}(t)$.
  \item Izračunajte i skicirajte trenutnu brzinu u trenucima $t_1=1\ s$, $t_2=2\ s$, $t_3=3\ s$ i $t_4=4\ s$.
  \item Odredite trenutno ubrzanje $\vec{a}(t)$ i skicirajte ga u nekoliko točaka putanje.
\end{enumerate}
