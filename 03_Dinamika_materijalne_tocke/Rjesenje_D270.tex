

$$ \vec{F}_R=m\vec{a}$$
Ako je brzina stalna akceleracija je jednaka nuli
$$\vec{F}+\vec{G}+\vec{R}+\vec{F}_{tr}=\vec{0} $$
Skalarno množimo s jediničnim vektorima $\vec{j}$ i $\vec{k}$
$$ \textbf{y:}\ \  F\cos\alpha -F_{tr}=0 $$
$$  \textbf{z:}\ \ F\sin\alpha-G + R =0$$

Iz $z$ smjera možemo izraziti silu reakcije podloge
$R=mg-F\sin\alpha$.
 
Za silu trenja pišemo $F_{tr}=\mu_k R=\mu_k (mg-F\sin\alpha)$ te uvrštavamo u izraz za $y$ smjer.
$$
F=\frac{mg\mu_k}{\cos\alpha+\mu_k\sin\alpha}
$$
Dobili smo izraz za silu u ovisnosti o kutu $\alpha$ ($F\Rightarrow F(\alpha)$), kako bismo dobili ekstrem funkcije potrebno je gornji izraz derivirati po kutu i izjednačiti s nulom. 
$$
\frac{d F(\alpha)}{d \alpha}= \frac{mg\mu_k}{(\cos\alpha+\mu_k\sin\alpha)^2}(-\sin\alpha+\mu_k\cos\alpha)=0
$$
Rješavanjem po kutu $\alpha$ dobivamo 
$$
\tan\alpha=\mu_k
$$
$$
\alpha= \arctan\mu_k=\arctan0,15=8,53^\circ
$$



