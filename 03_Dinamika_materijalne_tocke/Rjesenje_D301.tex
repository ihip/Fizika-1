
Iznos sile kojom blok A djeluje na blok B jednaka je iznosu sile kojom blok B djeluje na blok A 

$T=|\vec{T}_{AB}|=|\vec{T}_{BA}|$.

\begin{enumerate}[label=\alph*)]
  \item Zapišemo sve sile koje djeluju na  $$\textbf{blok B:}\ \ \ \vec{T}_{AB} + \vec{G}_B+\vec{R}_B=m_B\vec{a}\ \ \ \ \  /\cdot\vec{j}/\cdot\vec{k}$$
  $$\textbf{blok A:}\ \ \ \vec{F}_0+ \vec{T}_{BA} + \vec{G}_A+\vec{R}_A=m_A\vec{a} \ \ \ \ \  /\cdot\vec{j}/\cdot\vec{k} $$
  
  Radimo projekciju sila za blok B na os $y$ i $z$ 
  $$ \textbf{B,z:} \ \ \ 0 - G_B+R_B=0   \ \ \Rightarrow \ \  R_B=G_B$$
   $$\textbf{B,y:}\ \ \ T_{AB}+0+0=m_Ba \ \ \Rightarrow \ \ T=m_B a $$
  Isto radimo za blok A:
  $$ \textbf{A,z:}\ \ \ 0+0+G_A+R_A=0 \ \ \Rightarrow \ \  R_A=G_A  $$
   $$ \textbf{A,y:}\ \ \ F_0-T_{BA}+ 0+ 0=m_A a \ \ \Rightarrow \ \   F_0-T=m_Aa $$
 U poslijednji izraz možemo zamjeniti napetost niti $T$ sa izrazom iz $\textbf{B,y}$ 
 $$ F_0-m_B a=m_Aa $$
   $$ m_A a + m_B a =F_0$$
   $$ a= \frac{F_0}{m_A+m_B}= \frac{50N}{5kg+3kg}=6,25\ ms^{-2}$$
$$ T=m_B a=3kg\cdot 6,25ms^{-2}=18,75\ N $$   
\item
Zapišemo sve sile koje djeluju na $$\textbf{blok A:}\ \ \ \vec{F}_0+ \vec{T}_{BA} + \vec{G}_A+\vec{R}_A + \vec{F}_{tr,A}=m_A\vec{a}  \ \ \ \ \  /\cdot\vec{j}/\cdot\vec{k}$$
$$\textbf{blok B:}\ \ \ \vec{T}_{AB} + \vec{G}_B+\vec{R}_B+\vec{F}_{tr,B}=m_B\vec{a}\ \ \ \ \  /\cdot\vec{j}/\cdot\vec{k}$$
 Radimo projekciju sila za blok A na os $y$ i $z$ 
  $$ \textbf{A,y:}\ \ \ F_0-T_{BA}+ 0+ 0-F_{tr,A}=m_A a \ \ \Rightarrow \ \   F_0-T-\mu_kR_A=m_Aa $$
 $$ \textbf{A,z:}\ \ \ 0+0+G_A+R_A+0=0 \ \ \Rightarrow \ \  R_A=G_A  $$
Dobivamo $F_0-T-\mu_kG_A=m_Aa $. Isto radimo za blok B:
$$\textbf{B,y:}\ \ \ T_{AB}+0+0-F_{tr,B}=m_Ba \ \ \Rightarrow \ \ T-\mu_kR_B=m_B a $$
$$ \textbf{B,z:} \ \ \ 0 - G_B+R_B=0   \ \ \Rightarrow \ \  R_B=G_B$$
Dobivamo $T=m_Ba+\mu_kG_B$.
$$F_0 -m_Ba-\mu_km_Bg -\mu_km_Ag=m_Aa$$
Posložimo i izrazimo ubrzanje
 $$F_0-\mu_k(m_A+m_B)g=(m_A+m_B)a  $$
$$a=\frac{F_0}{m_A+m_B} -\mu_kg $$ 
$$ a=\frac{50N}{5kg+3kg} -0,3\cdot9,81ms^{-2}=3,307\ ms^{-2}$$

Još moramo izračunati napetost niti
$$ T= m_B(a+\mu_kg) $$ Ubrzanje možemo zamjeniti s dobivenim izrazom
$$ T= m_B(\frac{F_0}{m_A+m_B} -\mu_kg+\mu_kg)=\frac{m_BF_0}{m_A+m_B} $$
$$T=18,75\ N $$
   \end{enumerate}
  
