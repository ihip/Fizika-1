
\noindent 
\textbf{\stepcounter{zadatak}
\thecjelina.\thezadatak.}
Prema Zemlji se iz velike ("beskonačne") udaljenosti početnom brzinom iznosa $v_0 =3\ kms^{-1}$
duž pravca koji prolazi njezinim središtem giba meteor. Koliki će biti iznos brzine
meteora u trenutku kada se meteor nađe na udaljenosti $r = 6R_Z$ od središta Zemlje? Što se
događa s njegovom brzinom u odnosu na početnu? Koji je razlog tome?

