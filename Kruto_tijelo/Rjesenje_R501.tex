
Kako bismo izračunali kinetičku energiju rotacije $E_k= \frac{1}{2}I\omega^2$ moramo znati moment tromosti oko osi rotacije i iznos kutne brzine. Kako bismo odredili moment tromosti koristimo teorem o paralelnim osima (Steinerov teorem):
$$ I=I_T+Md^2 $$
gdje je $I_T$ moment tromosti oko osi koja prolazi kroz centar mase i za valjak iznosi $I_T=\frac{1}{2}MR^2$, $M$ je u ovom slučaju masa valjka, a $d$ je udaljenost između osi koja prolazi centrom mase i osi rotacije. Tako da moment tromosti možemo pisati 
$$I=\frac{1}{2}MR^2+MR^2=\frac{3}{2}MR^2. $$
Masu valjka možemo izraziti preko gustoće i volumena valjka ($V=R^2\pi h$), 
$$I=\frac{3}{2}\pi\rho h  R^4 .$$
Ostalo je izračunati kutnu brzinu koja je broj okretaja u sekunti puta $2\pi$
$$\omega=\frac{105}{60\ s}2\pi\ rad=10,995\ rads^{-1}\simeq11\ rads^{-1} $$.
Sada možemo izračunati kinetičku energiju rotacije:
$$E_k= \frac{3}{4}\pi\rho h  R^4 \omega^2 =\frac{3}{4}\pi 2700\ kgm^{-3}(0,08\ m)^4 0,32\ m (11\  rads^{-1})^2
$$

$$E_k=10,0895\ J $$
