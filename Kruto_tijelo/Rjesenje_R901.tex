 U početnom trenutku uteg mase $m$ ima potencijalnu energiju u polju sile 
teže $E_{p,G}(A)=mgh$. Neposredno prije udara o tlo uteg ima kinetičku energiju $E_k(B)=\frac{mv^2}{2}$ i valjak se zavrtio 
kutnom brzinom $\omega$ te ima kinetičku energiju rotacije $E_{k,R}(B)=\frac{I\omega^2}{2}$. Iskoristimo zakon očuvanja energije:

$$E_p(B)+E_k(B)=E_p(A)+E_k(A) $$
$$ 0+ \frac{mv^2}{2} + \frac{I\omega^2 }{2}= mgh + 0 .$$
Moment tromosti valjka koji rotira oko svoje osi iznosi $I=\frac{MR^2}{2}$. Također, obodna brzina 
ruba valjaka jednaka je brzini kojom uteg pada:
$$v=\omega R. $$
Ddobivamo:
$$ \omega= \sqrt{\frac{4mgh}{R^2(2m+M)}}. $$


\textit{Isto rješenje, drugi pristup.}

Zadatak je moguće riješiti pomoću jednadžbi gibanja. Kod rotacije krutog tijela moment sile jednak je produktu momenta tromosti i kutnog ubrzanja $N=I\alpha$. Budući da sila napetosti niti $T$ djeluje na obodu valjka, krak sile je jednak polumjeru utega $N=RT$. Za uteg na koji djeluju sila teža $G$ i napetost niti $T$ pišemo drugi Newtonov zakon $G-T=ma$. Sve zajedno dobivamo:

$$R(G-ma)=I\alpha.$$

Brzina utega jednaka je obodnoj brzini ruba valjka  $v=\omega R $, isto vrijedi i za ubrzanje $a=\alpha R$. Uvrštavanjem u gornji izraz dobivamo: 
$$ \alpha(I+mR^2)=RG,$$

$$ \alpha=\frac{Rmg}{I+mR^2}=konst.$$

Za konstantno ubrzanja vrijedi $\omega=\alpha t$ i $\varphi=\frac{1}{2}\alpha t^2 $. Kut $\varphi$ ovisit će o visini s koje pada uteg
$\varphi R=h$, ako to iskoristimo dobivamo:

$$t=\sqrt{\frac{2h}{\alpha R}}. $$

Na kraju se dobije:

$$\omega=\alpha t=\frac{Rmg}{I+mR^2} \sqrt{\frac{2h}{\alpha R}}=  \sqrt{\frac{4mgh}{R^2(2m+M)}}. $$
