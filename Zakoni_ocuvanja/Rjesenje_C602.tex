

Pišemo zakon očuvanja energije
$$ E_k(B) + E_{p,G}(B)  = E_k(A) + E_{p,G}(A) + W_{AB}$$
$$ \frac{1}{2}mv^2 + 0 = 0 + mgH + \vec{F}_{tr}\cdot\Delta\vec{r}  $$
Ostalo je za izračunati rad sile trenja
$$\vec{F}_{tr}\cdot\Delta\vec{r}=|\vec{F}_{tr}||\Delta\vec{r}|\cos \sphericalangle (\vec{F}_{tr},\Delta\vec{r})
=F_{tr}\Delta r\cos(\pi)$$
Pomak tijela $\Delta r$ možemo izaraziti iz visine kosine i kuta $\Delta r=H/\sin\vartheta$. Potrebno je 
još zapisati silu trenja koja ovisi o kinematičkom koeficijentu trenja i sili kojom tijelo pritišće podlogu
$F_{tr} = \mu_kmg\cos\vartheta $.
$$\vec{F}_{tr}\cdot\Delta\vec{r}= - \mu_kmg\cos\vartheta\frac{H}{\sin\vartheta}= - \mu_k mgH\cot\vartheta$$
Vraćamo se u zakon očuvanja energije
$$ \frac{1}{2}mv^2 = mgH - \mu_k mgH\cot\vartheta  $$
$$ v=\sqrt{2gH(1 - \mu_k\cot\vartheta)}$$
$$ v = \sqrt{2 \cdot 9,81 \ ms^{-2} \cdot 0,8 \ m (1-0,43\cdot\cot30^\circ)}=2,0 \ ms^{-1} $$


