\Oznaka{}{Mikuličić10:1.119.,str.46-modificirani}{}
%\Komentar{}
\Rokovi{2015-L6, 2016-L7, 2017-L7, 2018-L7, 2019-L7, 2020-L7}
\Tekst{Ledolomac mase $6000$ tona s ugašenim motorom nalijeće brzinom $30\ kmh^{-1}$ na santu leda koja se giba brzinom $2\ kmh^{-1}$ 
u istom smjeru. Poslije sudara zajedno se kreću brzinom $5\ kmh^{-1}$. Kolika je masa sante leda?


}
\Rjesenje{$m=50 000\ tona$}
\Postupak{
Zapisujemo zakona očuvanja količine gibanja i izražavamo masu sante leda

$$ m_1v_1 + m_2v_2=(m_1 + m_2)v'$$
$$m_2v_2 - m_2v' = m_1v' - m_1v_1 $$
$$ m_2 = \frac{v' - v_1}{v_2 - v'}m_1 $$
$$ m_2 = \frac{5\ kmh^{-1} - 30\ kmh^{-1} }{2\ kmh^{-1} - 5 \ kmh^{-1}}6000 \ t =50 000\ t $$

}
