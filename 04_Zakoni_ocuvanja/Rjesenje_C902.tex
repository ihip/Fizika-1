
Prije početka gibanja klizač miruje zajedno s predmetom $v'=0 $ stoga možemo izraziti
iz zakona očuvanja količine gibanja brzinu klizača na početku njegovog gibanja
$$ (m_1+m_2)v'=m_1v_1+m_2v_2  $$
$$ 0 = m_1v_1+m_2v_2  \ \ \Rightarrow \ \  v_1=-\frac{m_2}{m_1} v_2$$
Zapisujemo zakon očuvanja energije za klizača 
$$  E_k(B) + E_{p}(B)  = E_k(A) + E_{p}(A) + W_{AB}.$$ 
Budući da nema promjene visine potencijalna energija klizača je jednaka nuli, a kako na kraju svojega gibanja staje njegova kinetička energija $E_k(B)$ će također biti jednaka nuli
$$ 0 + 0 = \frac{1}{2}mv_1^2 + 0 + \vec{F}_{tr}\cdot\Delta\vec{r}$$
$$  0 = \frac{1}{2}mv_1^2 + F_{tr} \Delta r\cos\sphericalangle(\vec{F}_{tr},\Delta\vec{r})  $$ 
$$  0 = \frac{1}{2}mv_1^2 + F_{tr} \Delta r\cos\pi  $$ 
$$  \Delta r= \frac{1}{2} \frac{v_1^2}{\mu_k g} = \frac{m_2^2v_2^2}{2\mu_k m_1^2 g} $$
$$  \Delta r = \frac{(3\ kg)^2\cdot (8\ ms^{-1})^2 }{2\cdot 0,02 \cdot (70\ kg)^2 \cdot 9,81\ ms^{-2}}=0,3\ m $$

