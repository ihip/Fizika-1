


Koristimo zakon očuvanja energije. Metak na površini Mjeseca ima gravitacijsku potencijalnu energiju i kinetiču energiju, 
kada se popne na visinu $h$ ima samo gravitacijsku potencijalnu energiju
$$ E_{p,g}(h=0) +  E_k(h=0) = E_{p,g}(h) +  E_k(h) $$
$$ -\gamma\frac{M_Mm}{R_M} + \frac{1}{2}mv_0^2  =  -\gamma\frac{M_Mm}{R_M+h} + 0 $$
$$ R_M+h = \frac{-\gamma M_M}{-\gamma\frac{M_Mm}{R_M} + \frac{1}{2}v_0^2  } $$
$$ h = \frac{ -2\gamma M_M R_M}{-2\gamma M_M + v_0^2R_M} - R_M  $$

$$ h = \frac{ -2\cdot6,67\cdot  10^{-11}\ Nm^2kg^{-2}   7,34\cdot10^{22}\ kg 1,737\cdot10^6\ m}
{-2\cdot6,67\cdot  10^{-11}\ Nm^2kg^{-2} 7,34\cdot10^{22}\ kg + (715\ ms^{-1})^2  1,737\cdot10^6\ m} - 1,737\cdot10^6\ m  $$
$$ h=173\ 239,9\  m  $$
