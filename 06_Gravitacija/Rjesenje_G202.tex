

Jakost gravitacijskog polja Zemlje na visini $h$ možemo zapisati
$$ G(h)= \gamma\frac{M_Z}{(R_Z+h)^2} . $$
Tražimo za koju visinu $h$ vrijedi $G(h)=0,3g.$
$$ \gamma\frac{M_Z}{(R_Z+h)^2} = 0,3g$$
$$ (R_Z+h)^2 = \frac{\gamma M_z}{0,3g} $$
$$ h = \sqrt{\gamma  \frac{M_z}{0,3g}}  - R_Z $$
$$ h = \sqrt{ 6,67\cdot  10^{-11}\ Nm^2kg^{-2}  \frac{ 5,98 \cdot  10^{24}\ kg }{0,3\cdot 9,81\ ms^{-2}} }  
- 6,371\cdot  10^6\ m$$

$$ h= 5,271\cdot 10^6\ m $$
